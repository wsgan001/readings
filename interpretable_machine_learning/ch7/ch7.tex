\documentclass[a4paper]{article}

\usepackage{fullpage} % Package to use full page
\usepackage{parskip} % Package to tweak paragraph skipping
\usepackage{tikz} % Package for drawing
\usepackage{amsmath}
\usepackage{hyperref}

\title{Chapter 7 - A Look into the Crystal Ball}
\date{}

\begin{document}

\maketitle

\section{Introduction}
Let us assume that all informatin will be be digitized, automatable tasks will
be automated, and that for interesting problems it is not possible to exactly
specify our goal with all possible constraints (a corporation tasked only
with maximizing profits might do crazy things like pollute rivers, hire child
laborers, and fund militias in developing countries to achieve this).

With this in hand, let's predict the future.

\section{The Future of Machine Learning}
Machine learning adoption will advance, slowly but surely. It takes time for
large organizations to figure out good team structures, good data architectures,
and good systems to enable data scientists.

Machine learning will become easier to use.

More and more tasks will be formulated as prediction problems so that machine
learning can be used.

As machine learning is used in high stakes tasks and regulations are created,
interpretability will become more important. Many people today do not use machine
learning simply because it is not interpretable.

\section{The Future of Interpretability}
Model-agnostic interpretability tools will dominate because they are the most
portable. But, intrinsically interpretable methods might have a use for some
cases.

Most systems will include automated interpretability in the same way they
include automatic hyperparameter selection and automatic ensembling.

People will be able to train machine learning models even without knowing how to
program.

People will use interpretable machine learning to glean insights about their
data, not just for using the model.

Traditional methods, like linear models, make too many unrealistic assumptions,
so black box models will become more widely used because they will achieve
better performance.

Traditional statistical techniques like hypothesis tests and confidence intervals
will be adapted for black box methods.

Data scientists will automate away many of their tasks.

Robots and programs will have user interfaces to explain why they make certain
decisions.

Interpretability might help us understand more about intelligence in general.



\end{document}
