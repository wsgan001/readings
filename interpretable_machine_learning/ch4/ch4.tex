\documentclass[a4paper]{article}

\usepackage{fullpage} % Package to use full page
\usepackage{parskip} % Package to tweak paragraph skipping
\usepackage{tikz} % Package for drawing
\usepackage{amsmath}
\usepackage{hyperref}

\title{Chapter 4 - Interpretable Models}
\date{}

\begin{document}

\maketitle

\section{Introduction}
Let's talk about how to interpret some specific models. Monotonicity means that
an increase in feature value always has a single impact on prediction (RuleFit,
$k$ nearest neighbors, and (sort of) decision trees are non-monotonic). RuleFit
and decision trees are the only models that can also account for interactions
between features.

\section{Linear Regression}
The prediction is $\hat{y} = \hat{\beta_0} + \hat{\beta_1} x_1 + ... +
\hat{\beta_p}
x_p$ where the weights $\hat{\beta_0}, ..., \hat{\beta_p}$ are learned by
minimizing squared error. You can construct confidence intervals around the
weights to measure uncertainty.

The linear model is a bad fit for your data if it is not linear, if noise is
not normally distributed, if the noise does not have constant variance, if
instances are not independent of each other, or features are strongly
correlated.

Increasing a numerical feature increases the prediction by its weight. For a
binary feature, changing from reference category (0) to the other category
adjusts the prediction by its weight. For a categorical feature with multiple
categories, you can use one-hot encoding (with $L - 1$ entries for $L$
categories - all $0$ is the reference category). The intercept $\beta_0$ is the
prediction when all categorical features are at the reference class and all
features are at 0 (so make sure to give numerical features zero mean and unit
variance).

To measure how well your model fits the data, use $R^2 = 1 - SSE / SST$ where $SSE
= \sum_{i=1}^{n}{(y^{(i)} - \hat{y}^{(i)})^2}$ and $SST = \sum_{i=1}^{n}{(y^{(i)} - \bar{y})^2}$. Actually, you should use $\bar{R}^2 = R^2 - (1 - R^2) \frac{p}{
n - p - 1}$ where $n$ and $p$ are the number of instances and features,
respectively. $R^2$ is between $0$ and $1$, where larger values indicate better
fit.

The importance of a feature is $t_{\hat{\beta_j}} = \frac{\hat{\beta_j}}{
SE(\hat{\beta_j})}$.

One visual interpretation is a weight plot, make the feature name the $y$ axis.
Make the feature weight the $x$ axis, and plot the feature weight (and
confidence interval) for each feature. In order to compare features, make sure
they all have zero mean and unit variance.

Another visual is the effect plot. $effect_j^{(i)} = w_j x^{(i)}_j$. Now
make the $y$ axis the feature name, $x$ axis the effect, and plot a box and
whiskers plot for each feature. Again, features should have
zero mean and unit variance. This visualization handles categorical variables
better than the weight plot. If you are considering an individual prediction,
you can compute its effects and overlay them on the corresponding box and
whiskers plots.

Suppose we have six examples that take on categories A, A, B, B, C and C,
respectively. How do we encode this? In treatment coding, A is the reference
category $1, 0, 0$, B is $1, 1, 0$, and C is $1, 0,1$. Notice that the first $1$
is fixed. In effect coding, we get A $1, -1, -1$, B $1, 1, 0$, and C $1, 0, 1$.
In dummy coding we get A $1, 0, 0$, B $0, 1, 0$, and C $0, 0, 1$.

Linear models are contrastive, not selective, simple, and general. If your data
is linear, they are truthful.

If you want to have a model that gives 0 weight to less important features, add
the L1 regularization term $\lambda ||\beta||_1$ to the least squares objective.
Pick lambda by putting feature weight on the $y$ axis, $\log{\lambda}$ on the
$x$ axis, and plotting each feature weight as you vary lambda. Pick the lambda
that gives you the desired number of nonzero features. Other options to get
sparsity is to use domain knowledge, pick features correlated with the target,
use forward selection (greedily add features that improve $R^2$ the most),
or use backward selection.

Linear models are simple, well studied, available everywhere, and efficient.
However, data is rarely linear so they don't perform well. They also struggle
when you have strongly correlated features.

\section{Logistic Regression}
Linear regression does not work well for classification because it does not
output probabilities, extrapolates, has no meaningful threshold value
between classes, and does not generalize to multiple classes. Logistic
regression overcomes these problems. For binary classification, logistic
regression computes the log odds with $\log{\frac{P(y = 1)}{P(y = 0)}} = \beta_0 +
\beta_1 x_1 + ... + \beta_p x_p$. You may find it easier to look at the odds
instead of the log odds. You can also solve for $P(y = 1)$ by noting that
$P(Y = 0) = 1 - P(Y = 1)$.

Logistic regression has the same pros and cons of linear regression. It is
more complicated than the linear regression model because of the log odds.
It is nice because it gives you a probability. You can generalize it to
multiclass classification (it is called multinomial classification in this
case).

\section{GLM, GAM and more}
Let's look at ways to extend the linear regression model. Linear regression
assumes the target is a linear function of the features plus Gaussian noise. If
the noise is not Gaussian (e.g. the target cannot be negative), use a
generalized linear model (GLM). If the features interact, add interactions
manually. If the relationship is not linear, use a generalized additive model
(GAM) or add feature transformations as features.

A GLM has $g(E_Y(y|x)) = \beta_0 + \beta_1 x_1 + ... + \beta_p x_p$, where
$g$ is the link function and $Y$ is a probabiblity distribution from the
exponential family. If the target is a count, use a Poisson distribution. If
the target is always positive, use the exponential distribution, The linear
regression model uses the Gaussian distribution. The logistic regression
(multinomial regression) model uses a Bernoulli (multinoulli) distribution. Each
distribution has a canonical link function (Gaussian has identity,
Bernoulli has $\ln$, Poisson has $\ln$), but you can pick your own.

If features interact, add interactions as features. For example, multiply two
numerical features. Or multiply numerical feature by each entry of one-hot
encoded of categorical feature. Or compute all possible combinations of two
categorical features and create a separate category for each and encode that.
This can yield a lot of interaction features if applied naively, so use an
algorithm like RuleFit to pick only the most promising interaction terms.

What if the underlying relationship between features and targets is not
linear? One option is to transform a feature (e.g. take its logarithm, square
root, or exponential) and add that as a new feature. Another option is to
discretize numerical features and make them into a categorical feature (this
is hard to do well though). Another option is to use a GAM, which models
$g(E_Y(y|x)) = \beta_0 + f_1(x_1) + ... + f_p(x_p)$. A good choice for the $f_j$
functions are splines, which are sums of curves (read about these elsewhere).

All these extensions to linear models can make them harder to interpret. There
are also tons of other variants of linear models (data not i.i.d = mixed models
or generalized estimating equations, noise is not heteroscedastic = robust
regression, outliers in data = robust regression, predicting time until
event occurs = parametric survival models, cox regression, survival analysis,
predicting a category = multinomial regression or logistic regression,
predicting ordered categories = proportional odds model, predicting a count
= Poisson regression, predicting a count where 0 is common = zero-inflated Poisson
regression, hurdle model, not sure what features you need = causal inference,
mediation analysis, missing data = multiple imputation, have prior knowledge
= Bayesian inference).

\section{Decision Tree}
Decision trees split based on feature values until they hit a leaf node, which
is where they make a prediction. CART is a popular algorithm for learning
decision trees. The model is $\hat{y} = \hat{f}(x) = \sum_{m=1}^{M}{c_m
I[x \in R_m]}$ where the leaf value (average of training examples in leaf) is
$c_m$ and $I[x \in R_m]$ indicates whether $x$ follows the path to leaf $m$.
CART greedily picks features and split points (or split
subsets in the case of categorical features) to minimize some measure of
heterogenaity in the leaves (variance for regression, Gini index for
classification). The algorithm continues until hitting a maximum number of nodes
or tree depth.

A decision tree is simply a flowchart, so it's easy to see how a prediction
is made.

To compute feature importance, look at all splits where feature is used
and sum up how much it reduces variance (or Gini index) - normalize importances
so that their sum is 100.

The predicted value of an internal node is the average of all instances in the
leaves under it. So, you can see how the predicted value is changing as you
go through the tree. You can also compute the amount that each feature
has contributed.

Decision trees are easily understood and can handle feature interactions. It
also makes counterfactual analysis easy (what if I went down this path
instead?). They also don't require you to standardize numerical features or
encode categorical features. Their key disadvantages are that they can only
approximate linear relationships, are not smooth, and are unstable (changing
the dataset may yield very different tree).

\section{Decision Rules}
These are IF-THEN rules (IF = condition/antecedent, THEN = prediction). These
rules can have ANDs or ORs in them. A rule is evaluated by its support (fraction
of examples for which the antecedent applies) and accuracy (when the antecedent
applies, what fraction of the time is it correct). When you have many rules,
some rules might overlap and there may be examples that do not fit any rules.
We fix overlap with a decision list (i.e. we consider rules one at a time) or
decision set (i.e. the rules run independently and we average their votes). We
also make a default rule that applies if no other rule matches. We can learn
rules with OneR, sequential covering, and Bayesian Rule Lists.

OneR first discretizes continuous features. Then for each feature we consider
each discrete value, look at dominant target value (if you are doing regression,
discretize the target), and create a rule and
compute the error of the rule. We then pick the feature with smallest overall
error. If features have many possible discrete values, OneR can overfit, so use
a training set to make rules and validation set to evaluate error.

Sequential covering makes a decision list (or set). We iteratively build up
the list (or set) by adding the rule to the list (or set) and removing points
that the rule covers. We keep doing this until our rules meet some quality
threshold. For multiclass settings, we treat the least common class as class 1
and all other classes as class 0. Once we finish learning class 1, we consider
the next least common class and repeat the process. To learn a rule, we fit
a decision tree and recursively select the purest nodes and turn the path from
node to leaf into a rule. There are other ways to learn rules. RIPPER is a
more sophisticated variant of sequential covering.

In Bayesian Rule Lists, we first pre-mine frequent patterns and then build
a rule list from them. To mine frequent patterns, we don't need the targets.
A pattern is just a set of (feature, feature value) pairs. The frequency is
the support. You can mine frequent patterns with the Apriori or FP-Growth
algorithms. Apriori finds frequent patterns and builds association rules, but we
only care about finding the frequent patterns. Apriori first finds single
(feature, value) pairs whose support exceeds a threshold. It then combines
these patterns with AND to look for higher order patterns with sufficiently
high support. With these in hand, we can use BRL to incorporate prior knowledge
that prefers short rules and a short list. We first generate an initial list
from the prior distribution, then we sample lists by adding/removing rules,
and finally we select the list with highest posterior probability. See the book
for details.

Decision lists are very easy to interpret, fast to run, and robust against
outliers. They also tend to produce sparse models, which is good. However, they
are not great for regression or continuous features because you need to
discretize. Except for RIPPER and BRL, they tend to overfit.

\section{RuleFit}
This learns sparse linear models that learns interaction effects (as decision
rules). Basically, we learn decision trees, pick paths from root to leaf as
decision rules (i.e. just AND the splits), and use these as features along with
the regular features, and do Lasso (L1) linear regression. You can interpret them
with the same techniques we use for linear models. The importance of a feature
should also include all the rules that it appears in.

Note that you can fit a
random forest or gradient boosted tree on the problem and then extract the
rules from the resulting model. Then, take your features and windsorize them
(i.e. constrain them between the 5th percentile and 95th percentile values).
Next, add linear features (one per windsorized feature) to the set of
decision rules. These linear features should be $0.4 * f / std(f)$ where $f$
is the feature - we do this so they are comparable to the decision rules. Then,
do Lasso (L1) regression.

To compute feature importance for the linear terms, we do $I_j = |\beta_j| std(
l_j(x_j))$ where $\beta_j$ is the learned weight and $l_j(x_j)$ is the linear
term. For decision rules we have
$I_k = |\alpha_k| \sqrt{s_k (1 - s_k)}$ where $\alpha_k$ is the learned weight
and $s_k$ is the support of the rule. The overall importance for a feature is
the importance of its linear term plus its importance in all the rules that it
appears in (a rule splits its importance evenly over the features inside it).

RuleFit is great because it mixes linear models and rule models. It also has
good local interpretability because most rules will not apply for a given data
point. It works well with methods like feature importance, partial dependence
plots, and feature interactions. RuleFit can be confusing when there are too
many rules or when there are overlapping rules.

\section{Other Interpretable Models}
Naive Bayes assumes features are independent and computes $P(C_k | x) = \frac{1
}{Z} P(C_k) \sum_{i=1}^{n}{P(x_i | C_k)}$ where $Z$ is a normalization constant
that ensures the probabilities sum to 1. This is interpretable because you can
see how much probability each feature contributes.

The $k$ nearest neighbors method classifies a point by averaging the predictions
for its $k$ nearest neighbors in the training set. It is kind of interpretable
at the local level because you can look at the neighbors.

\end{document}
