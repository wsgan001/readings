\documentclass[a4paper]{article}

\usepackage{fullpage} % Package to use full page
\usepackage{parskip} % Package to tweak paragraph skipping
\usepackage{tikz} % Package for drawing
\usepackage{amsmath}
\usepackage{hyperref}

\title{Chapter 1 - Introduction}
\date{}

\begin{document}

\maketitle

\section{Introduction}
This book describes how to make supervised machine learning models
interpretable. Basically, you can use interpretable models or model-agnostic
methods (can give local or global explanations).

\section{Story Time}
Three fictional stories that are supposed to motivate why interpretability is
important. Skip.

\section{What is Machine Learning?}
We focus on supervised learning (not unsupervised or reinforcement learning).
Machine learning often beats hand designed algorithms, but is hard to debug.

\section{Terminology}
An algorithm is a set of rules. Machine learning is a set of techniques for
learning from data and making predictions. A machine learning algorithm
defines how to create a machine learning model from data. A machine learning
model maps inputs to outputs. A black box (in contrast to white box) model is
one whose internal mechanisms are not decipherable (e.g. a complicated
neural net is a black box model, a simple linear regression is white box).

Interpretable machine learning is a set of techniques that aims to make a
model's predictions understandable to humans.

A dataset is made of (features, target) pairs. We assume features are
interpretable. A prediction is a model's estimate of the target.

\end{document}
