\documentclass[a4paper]{article}

\usepackage{fullpage} % Package to use full page
\usepackage{parskip} % Package to tweak paragraph skipping
\usepackage{tikz} % Package for drawing
\usepackage{amsmath}
\usepackage{hyperref}

\title{Chapter 3 - Linear Models for Outlier Detection}
\date{}

\begin{document}

\maketitle

\section{Introduction}
We can predict one variable (feature) from the others using linear regression
or project the data to a lower dimensional linear subspace with Principal
Components Analysis (PCA).
\section{Linear Regression Models}
\begin{align}
  y = \sum_{i=1}^{d}{w_i x_i} + w_{d+1}
\end{align}

This is a linear model. We learn the weights by minimizing the sum of squared
errors. Outliers are points whose squared error is large. If you put the
data into design matrix $D$, the weights into $\bar{W}$, and the regression
targets into $\bar{y}$, you get $\bar{y} \approx D \bar{W}^T$. Minimizing
$||D \bar{W}^T - \bar{y}||^2$ gives $\bar{W}^T = (D^T D)^{-1}D^T \bar{y}$.
If $D^T D$ is not invertible, we can regularize with parameter $\alpha$ and
minimize $||D \bar{W}^T - \bar{y}||^2 + \alpha ||\bar{W}||^2$ to get
$\bar{W}^T = (D^T D + \alpha I)^{-1}D^T \bar{y}$.

Ironically, the presence of outliers can mess up the inference of the parameters,
so it is typical to randomly sample some points and fit those. We do this
many times and create an ensemble. For a given point, we average the squared
error over each ensemble component to get the point's outlier score. We can also
use extreme value analysis on these squared errors.

To use linear regression in an unsupervised setting, we can consider each
feature one at a time, make it a dependent variable, and fit a model. Then,
we can average over all the models to get the outlier score for each point.

\section{Principal Components Analysis}
Linear regression projects $d$-dimensional data to a $(d-1)$-dimensional
hyperplane. PCA lets us project to a $k$-dimensional hyperplane.

The covariance is $\Sigma = \frac{D^T D}{N}$ and can be diagonalized as
$\Sigma = P \triangle P^T$ where $\triangle$ is the diagonal eigenvalue matrix
(the eigenvalue measures the variance on each dimension in the hyperplane)
and $P$ is the eigenvector matrix. We can project the data to the eigenvector
space with $D' = DP$. When we project, we should look at the dimensions
corresponding to the small eigenvalues because these are the dimensions with
maximum variance. Any data point that is extreme on these dimensions is an
outlier. More generally, we can remove the $k$ largest eigenvalue-eigenvector
dimensions and measure the reconstruction cost (the points with highest
reconstruction cost are outliers). Alternatively, we can compute the outlier
score in a soft way (where $\lambda_j$ and $\bar{e}_j$ are the eigenvalue,
eigenvector pair).

\begin{align}
  Score(\bar{X}) = \sum_{j=1}^{d}{\frac{|(\bar{X} - \bar{\mu})^T \bar{e}_j|^2}{
  \lambda_j}}
\end{align}

PCA can handle a few outliers. If there are many outliers, you can run PCA to
identify the outliers, remove them, run PCA again, and repeat. Before you
use PCA, make sure to standardize $D$ so each feature has zero mean and unit
variance. If there is not much data, we can regularize by replacing $\Sigma$
with $\Sigma + \alpha I$.

How do you pick the number of eigenvalues, $k$? You can use a $t$ test and
find outliers in the set of eigenvalues. It is typical that most eigenvalues
are small and there are a few big ones.

To make PCA nonlinear, we use the kernel trick. First, we note that we can
do PCA with similarity matrix $S = D D^T$ instead of $\Sigma$. Then, we
replace $S$ with a similarity matrix where $S_{ij} =
K(\bar{X}_i, \bar{X}_j)$ (we need to make sure to use a valid kernel function).
We then diagonalize as follows: $S = Q \Lambda^2 Q^T$ and pick $k$
eigenvectors. Typically we pick $k$ to get all nonzero eigenvalues because we
are looking for outlier behavior, not aggregate trends. This gives us the
embedding $D'$. We divide each column of $D'$ by its standard deviation and
report its squared distance from the centroid of $D'$ as its outlier score.

Popular kernels are the linear kernel $\bar{X}_i^T \bar{X}_j$,
Gaussian Radial Basis Function (RBF)
$\exp{(-\frac{||\bar{X}_i - \bar{X}_j||^2}{\sigma^2})}$, polynomial kernel
$(\bar{X}_i^T \bar{X}_j + c)^h$, and sigmoid kernel $\tanh{(\kappa \bar{X}_i^T
\bar{X}_j - \delta)}$. The RBF kernel is the most popular. For the RBF kernel,
use the median pairwise distance between points (or three times the pairwise
distance if you have fewer than 1000 data points) as the kernel
parameter. To speed up kernel
computation, you can simply set similarity to 0 if points are not $k$
nearest neighbors. There are kernel functions for complex data types like
strings, time series, and graphs.

The similarity matrix is $N \times N$, which is huge. So, we sample $s$ points
and use the technique described above to get $Q_k \Lambda_k$. We then compute
the similarity of the other points to the sample points to get
$(N - s) \times s$ matrix $S_o$. The embedding is then
$V_k = S_o Q_k \Lambda_k^{-1}$. We stack $Q_k \Lambda_k$ vertically on top
of $S_o Q_k \Lambda_k^{-1}$ to get the $N \times k$ embedding $E$. We
standardize each column to zero mean and unit variance (this means the mean
row vector is the zero vector). We then compute the squared distance of each
row to the mean row vector (0) as the outlier score for the data point at that
row.

\section{One-Class Support Vector Machines}
We assume that the origin in the kernel transformed space belongs to the outlier
class. Our decision boundary is thus: $\bar{W}^T \Phi(\bar{X}) - b = 0$. We then
minimize $J = \frac{1}{2} ||\bar{W}||^2 + \frac{C}{N} \sum_{i=1}^{N}{\max{(b - \bar{W}^T \Phi(\bar{X}_i), 0)} - b}$.

To support kernel functions, we rewrite $\bar{W}^T \Phi(\bar{Y}) - b =
\sum_{i=1}^{N}{\alpha_i K(\bar{Y}, \bar{X}_i)} - b$. We then minimize (where
$S$ is the similarity matrix) $\frac{1}{2} \bar{\alpha}^T S \bar{\alpha}$ subject
to $0 \leq \alpha_i \leq \frac{C}{N}$ for $i \in \{1...N\}$ and $\sum_{i=1}^{N}{
\alpha_i} = 1$. Solving the above with quadratic programming yields $\bar{\alpha}$
and we compute $b = \sum_{i=1}^{N}{\alpha_i K(\bar{Y}, \bar{X}_i)}$ where
$\bar{Y}$ is a support vector (this is a point where $0 < \alpha_j < C/N$). The
score of a point is then $Score(\bar{X}) = \sum_{i=1}^{N}{\alpha_i K(\bar{X},
\bar{X}_i)} - b$.

We can optimize the the quadratic programming problem with gradient descent
($\bar{\alpha} \gets \bar{\alpha} - \eta S \bar{\alpha}$, constrain each
$\alpha_i$ to $[0, C/N]$, and scale $\bar{\alpha}$ so that $\sum_{i=1}^{N}{
\alpha_i} = 1$).

One-class SVM is bad because of the origin assumption and scaling difficulties.
You can mitigate the former with binary features and the latter by sampling
a set of points and fitting them instead of fitting the whole dataset (you
can ensemble this). Use kernel PCA instead.

\section{A Matrix Factorization View of Linear Models}
From PCA, we get $D' = DP_k$. If we do PCA the other way, we get $DP_k \approx
Q_k \Lambda_k$ and thus $D \approx Q_k \Lambda_k P_k^T$. Combining the
diagonal matrix $\Lambda_k$ into one of the other factors means we have
decomposed $D \approx UV^T$ where $U$ and $V$ are low rank. Another way to find
the matrices is by minimizing the Frobenius norm $||D - UV^T||$ subject to
the columns of $U$ (and $V$) having mutually orthogonal (orthonormal) columns.
We can remove the constraints if we want. You can also add different
constraints.

What do we do if the data is incomplete? Assume we have $N$ users and $d$ movies
and we represent user-movie ratings with $N \times d$ matrix $D$, where the
element at $(i, j)$ is $x_{ij}$. Let $H = \{(i, j) : x_{ij} \textrm{ is not
missing}\}$. Decomposing $D$ into $UV^T$ means the predicted value for $(i, j)$
is $\sum_{s=1}^{k}{u_{is} v_{sj}}$. We then minimize $J = \frac{1}{2}
\sum_{(i, j) \in H}{(x_{ij} - \sum_{s=1}^{k}{u_{is} v_{sj}})^2}
+ \frac{\alpha}{2}(||U||^2 + ||V||^2)$. Letting $e_{ij} = (x_{ij} -
\sum_{s=1}^{k}{u_{is} v_{sj}})^2$ and computing the gradient gives us the
gradient descent update equations $U \gets U(1- \alpha \eta) + \eta EV$
and $V \gets V(1 - \alpha \eta) + \eta E^T U$. Here, $E$ is a matrix that has
the values in $H$, but is $0$ where $H$ is not defined. Letting $n_i$ be the
number of observed entries in row $i$ of $\bar{X}_i$, the outlier score is
then $Score(\bar{X}_i) = \frac{1}{n_i} \sum_{j:(i, j) \in H}{e_{ij}^2}$.


\section{Neural Networks: From Linear Models to Deep Learning}
A feedfoward neural network is typically a sequence of layers of the form
$z = \Phi(\bar{W}^T \bar{X} + b)$ where $\Phi$ is an activation function like
the sigmoid.

A one-class neural network aims to produce an output of $z = 0$ for any input
and nonzero weights. We aim to minimize $z^2$. We can do this with gradient
descent (and we force $\bar{W}$ to have unit norm). The outlier score is
then just the $z^2$ value. To avoid overfitting, we can split the data into
two parts. We train on part 1 and score part 2. Then we train on part 2 and
score part 1. We can ensemble this. If the neural network is a perceptron
(i.e. single layer with linear activation $\Phi(X) = X$), then this is the
same as a linear model. We can add many layers and different activation
functions and train with backpropagation.

A better alternative to one-class neural networks is replicator networks
(also called autoencoders), that simply aim to reconstruct ($x'$) the data $x$
and have a bottleneck layer (i.e. the layer's output dimensionality is less
than that of $x$). We minimize $\sum_{i=1}^{d}{(x_i - x'_i)^2}$. This is
more powerful than matrix factorization because it's nonlinear. The outlier
score is the reconstruction error. Pretraining (e.g. greedily training layers
one at a time) can help the model avoid local minima. When training a neural
net, don't make it too large (you'll overfit) and train on a random subset
of the data to reduce outlier impact.

\section{Limitations of Linear Modeling}
If the features are not correlated and do not fall onto a lower dimensional
manifold, it's hard to use linear modeling. Linear models are good to
ensemble with proximity based models (chapter 4). Linear models overfit when
your dataset is small. Linear models are not easily interpretable.

\section{Conclusions and Summary}
When features are correlated, linear models can remove outliers. PCA is the
best approach to try and can be extended to be nonlinear. Other methods
like SVM, matrix factorization, and neural networks can also help.

\end{document}
